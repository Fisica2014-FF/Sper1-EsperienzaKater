\documentclass[111pt,a4paper]{article} % Prepara un documento con un font grande



%----------------------------------------------------------------------------------------
%	PACKAGES AND OTHER DOCUMENT CONFIGURATIONS
%----------------------------------------------------------------------------------------


% Adatta LaTeX alle convenzioni tipografiche italiane,
% e ridefinisce alcuni titoli in italiano, come "Capitolo" al posto di "Chapter",
% se il documento è in italiano
\usepackage[italian]{babel}


\usepackage[utf8]{inputenc} % Consente l'uso caratteri accentati italiani

% Certe cose prese da http://www.latextemplates.com/template/journal-article
\usepackage[sc]{mathpazo} % Use the Palatino font
\linespread{1.05} % Line spacing - Palatino needs more space between lines
\usepackage{microtype} % Slightly tweak font spacing for aesthetics
\usepackage{booktabs} % Horizontal rules in tables
\usepackage[hang, small,labelfont=bf,up,textfont=it,up]{caption} % Custom
% captions under/above floats in tables or figures

\usepackage{amsmath} % Package matematico

\usepackage{graphicx}		% Per le immagini
\usepackage{gnuplot-lua-tikz}
%\usepackage[top=2.5cm, bottom=2cm, left=2cm, right=2cm]{geometry} % Il mio
% geometry originale
\usepackage[hmarginratio=1:1,top=32mm,columnsep=20pt]{geometry} % Document
% margins


%--------------------------------------------------------------------------
% Rinnova i comandi

\usepackage{titlesec} % Allows customization of titles
\renewcommand\thesection{\Roman{section}} % Roman numerals for the sections
\renewcommand\thesubsection{\Roman{subsection}} % Roman numerals for subsections
\titleformat{\section}[block]{\large\scshape\centering}{\thesection.}{1em}{} % Change the look of the section titles
\titleformat{\subsection}[block]{\large}{\thesubsection.}{1em}{} % Change the look of the section titles


%\nonstopmode %non fermarti agli errori

%\usepackage{fancyhdr}
%\setlength{\headheight}{15.2pt}
%\pagestyle{fancy} % Solo le pagine normali, non i titoli nè la pagina iniziale



\usepackage{fancyhdr} % Headers and footers
\pagestyle{fancy} % All pages have headers and footers
	\renewcommand{\sectionmark}[1]{ \markright{#1}{} } % Preso da http://en.wikibooks.org/wiki/LaTeX/Page_Layout#Customizing_with_fancyhdr
    \fancyhf{}% Resetta la formattazione precedente
\fancyhead[C]{Forcher, Piccinin - \emph{Volano} $\bullet$ \thesection.\
\rightmark}
% Custom header text
\fancyfoot[L]{\thepage} % Custom footer text
\fancyfoot[R]{Sec. \thesection} % Custom footer text


\DeclareGraphicsExtensions{.pdf, .png, .jpg} % Se due immagini hanno lo stesso nome sceglile secondo l'ordine di filetype qui
\graphicspath{ {./img/} }					 % Path delle immagini 

%\input{./preamboli_e_stili/titolo_Kater.tex}
%%%%%%%%%%%%%%%%%%%%%%%%%%%%%%%%%%%%%%5%%%%%%%%%%%%%%%%%%%%%%%%%%%%%%%%%%%%%%%%%%%
\usepackage{float}
\usepackage{caption}
%\usepackage{multirow}
%\usepackage[top=3.6cm, bottom=1.5in, left=0.5in, right=0.5in]{geometry}


% I miei stili di float, con le righe
\floatstyle{ruled}
\newfloat{tabella}{H}{lop}
\floatname{tabella}{Tabella}

\floatstyle{ruled}
\newfloat{grafico}{H}{lop}
\floatname{grafico}{Grafico}
%%%%%%%%%%%%%%%%%%%%%%%%%%%%%%%%%%%%%%%%%%%%%%%%%%%%%%%%%%%%%%%%%%%%%5%%%%%%%%%%%%%



%////////////////////////////////////////////////////////////////////////////////////////////////////////////////////////////
%////////////////////////////////////////////////////////////////////////////////////////////////////////////////////////////
% Fine dei dati iniziali per il latex: il documento finale inizierà da qui
\begin{document}

\maketitle % Produce il titolo a partire dai comandi \title, \author e \date
\tableofcontents % Prepara l'indice generale

% Le varie sezioni
%\section{Obiettivi}
\begin{abstract}
	L'obiettivo dell'esperienza è quello di stimare il momento di inerzia, I\ped{0}, di un corpo rigido, il volano, partendo dalle le accelerazioni angolari, $\alpha$ e $\beta$, e dal suo momento d'attrito, M\ped{att}.

\end{abstract}

\section{Descrizione dell'apparato strumentale}
	In quest'esperienza si è utilizzato, come apparato sperimentale, un volano
(volano numero 7F). Esso è composto da un disco (di raggio $R = 0.01895 \pm
0.00001 Kg$ forma complessa) di una certa massa appeso al muro tramite un perno, e
collegato tramite un filo di massa trascurabile a un pesetto di massa $34 \pm 0.5 g$, di modo che quando il
pesetto scenda il volano possa acquistare velocità angolare.
Il filo è legato al disco interno tramite un foro in quest'ultimo, in cui
viene incastrata l'altra estremità del filo in questione, il quale, una volta
spiegato per intero, viene rimosso, per mezzo della forza peso, lasciando così girare "a vuoto" il volano.



\section{Metodologia di misura}
	Si è inserito e avvolto il filo nel perno, facendo attenzione a non creare nodi, che potrebbero influire negativamente sulle misure.
Identificata una tacca sul volano per contare i giri, si è posizionato l'apparato e si è lasciato andare il peso. 
Uno di noi guardava il volano, con un dito sul pulsante del cronometro, mentre l'altro trascriveva i dati al computer. Dopo dodici giri il pesetto si staccava e il volano decelerava:
si lasciava trascorrere qualche giro mentre si resettava il cronometro (per non avere problemi di covarianza tra accelerazione e decelerazione) 
e si preparava il nuovo file di dati. Poi si ricominciava a misurare i tempi, abbiamo proceduto contando singoli giri perchè ci siamo resi conto
che altrimenti si perdeva facilmente il conto dei giri.
Essendo la decelerazione minore dell'accelerazione, in valore assoluto, si è
deciso di prendere dieci campioni ripetuti di 22 misure ciascuno. 


\newpage
\section{Presentazione dei dati}			
	\subsection{Tabelle}
	Sono risultati $\bar{\alpha} = 0.0206775 \pm 1.94669 \cdot 10^{-5}$, e
$\bar{\beta} = -0.00240092 \pm 2.2231 \cdot 10^{-5}$, che, dato che appaiono un
po' bassi, forse indicano qualche errore nel prendere i dati.

\begin{tabella}
    \centering
\begin{tabular}{ r r l|l r r }

\multicolumn{1}{l}{\alpha} & \multicolumn{1}{l}{\sigma \ped{alpha}} &  &  & \multicolumn{1}{l}{\beta} & \multicolumn{1}{l}{\sigma \ped{beta}} \\ \hline
0.02093 & 0.00007 &  &  & -0.005 & 0.001 \\ \hline
0.02080 & 0.00003 &  &  & -0.0023 & 0.0002 \\ \hline
0.02026 & 0.00007 &  &  & -0.0020 & 0.0004 \\ \hline
0.02064 & 0.00009 &  &  & -0.0029 & 0.0005 \\ \hline
0.0208 & 0.0001 &  &  & -0.0027 & 0.0005 \\ \hline
0.02063 & 0.00006 &  &  & -0.0020 & 0.0006 \\ \hline
0.0208 & 0.0001 &  &  & -0.0032 & 0.0005 \\ \hline
0.0210 & 0.0004 &  &  & -0.0034 & 0.0004 \\ \hline
0.02064 & 0.00005 &  &  & -0.0026 & 0.0005 \\ \hline
0.02052 & 0.00004 &  &  & -0.0015 & 0.0005 \\ 
\end{tabular}


\caption{Accelerazioni e decelerazioni risultanti}
\label{tab:dec}
\end{tabella}

\begin{tabella}
    \centering
\begin{tabular}{r r r r r r r r r r}

\multicolumn{ 10}{c}{Misure ripetute Tempi Accelerazione, $\pm 0.1$  [s]} \\ \hline
16.8 & 16.9 & 15.7 & 15.9 & 15.6 & 16.0 & 16.8 & 16.1 & 15.8 & 16.3 \\ \hline
24.2 & 24.0 & 22.8 & 23.4 & 23.0 & 23.2 & 23.9 & 23.2 & 23.0 & 23.6 \\ \hline
29.8 & 29.6 & 28.3 & 28.8 & 28.4 & 28.8 & 29.5 & 28.7 & 28.6 & 29.2 \\ \hline
34.6 & 34.2 & 32.9 & 33.5 & 33.1 & 33.4 & 34.1 & 33.3 & 33.3 & 33.9 \\ \hline
38.7 & 38.3 & 37.1 & 37.7 & 37.2 & 37.5 & 38.3 & 37.5 & 37.3 & 38.0 \\ \hline
42.4 & 42.0 & 40.6 & 41.4 & 40.8 & 41.3 & 42.3 & 41.1 & 41.0 & 41.7 \\ \hline
45.9 & 45.5 & 44.1 & 44.8 & 44.3 & 44.7 & 45.4 & 44.3 & 44.5 & 45.1 \\ \hline
49.1 & 48.6 & 47.2 & 47.9 & 47.3 & 47.8 & 48.5 & 46.4 & 47.6 & 48.3 \\ \hline
52.1 & 51.6 & 50.3 & 51.0 & 50.5 & 50.8 & 51.5 & 50.7 & 50.7 & 51.3 \\ \hline
54.8 & 54.5 & 53.0 & 53.6 & 53.3 & 53.8 & 54.3 & 53.5 & 53.4 & 54.2 \\ \hline
57.8 & 57.2 & 55.7 & 56.4 & 56.0 & 56.3 & 57.1 & 56.3 & 56.2 & 56.9 \\ \hline
60.3 & 59.6 & 58.4 & 59.0 & 58.3 & 59.0 & 59.6 & 58.8 & 58.8 & 59.4 \\ 
\end{tabular}


\caption{Tempi in accelerazione}
\label{tab:acc}
\end{tabella}

\begin{tabella}
    \centering
\begin{tabular}{r r r r r r r r r r}

\multicolumn{ 10}{c}{Misure ripetute Tempi Decelerazione, $\pm 0.1$  [s]} \\ \hline
2.4 & 2.5 & 2.6 & 2.8 & 2.5 & 2.6 & 2.5 & 2.6 & 2.6 & 2.6 \\ \hline
5.1 & 5.1 & 5.2 & 5.5 & 5.1 & 5.1 & 5.1 & 5.2 & 5.1 & 5.1 \\ \hline
7.7 & 7.7 & 7.8 & 8.2 & 7.7 & 7.6 & 7.7 & 7.9 & 7.7 & 7.7 \\ \hline
10.2 & 10.3 & 10.6 & 11.1 & 10.2 & 10.2 & 10.3 & 10.5 & 10.4 & 10.2 \\ \hline
12.9 & 12.8 & 13.1 & 13.9 & 12.9 & 12.9 & 13.0 & 13.1 & 13.1 & 12.8 \\ \hline
15.5 & 15.4 & 15.7 & 16.7 & 15.4 & 15.5 & 15.6 & 15.9 & 15.7 & 15.4 \\ \hline
18.1 & 18.1 & 18.4 & 19.5 & 18.1 & 18.1 & 18.2 & 18.5 & 18.3 & 18.1 \\ \hline
20.8 & 20.8 & 21.1 & 22.3 & 20.7 & 20.8 & 20.9 & 21.6 & 21.1 & 20.8 \\ \hline
23.4 & 23.3 & 23.8 & 25.2 & 23.3 & 23.4 & 23.5 & 24.1 & 23.6 & 23.4 \\ \hline
26.0 & 26.0 & 26.4 & 28.1 & 25.9 & 26.1 & 26.2 & 26.9 & 26.4 & 25.9 \\ \hline
28.7 & 28.7 & 29.2 & 30.9 & 28.6 & 28.7 & 28.8 & 29.7 & 29.1 & 28.6 \\ \hline
31.5 & 31.4 & 32.0 & 33.9 & 31.4 & 31.4 & 31.5 & 32.5 & 31.9 & 31.4 \\ \hline
34.3 & 34.2 & 34.8 & 36.9 & 34.0 & 34.2 & 33.6 & 35.3 & 34.5 & 34.0 \\ \hline
36.9 & 36.9 & 37.6 & 39.9 & 36.5 & 36.8 & 35.5 & 38.1 & 37.1 & 36.8 \\ \hline
39.7 & 39.6 & 40.3 & 42.8 & 39.4 & 39.6 & 39.8 & 40.8 & 40.1 & 39.6 \\ \hline
42.4 & 42.3 & 43.1 & 45.9 & 42.1 & 42.3 & 42.6 & 43.9 & 43.1 & 42.2 \\ \hline
45.3 & 45.1 & 45.9 & 48.9 & 44.8 & 45.1 & 45.4 & 46.7 & 45.6 & 45.1 \\ \hline
48.1 & 47.9 & 48.8 & 51.9 & 47.8 & 48.0 & 48.2 & 49.6 & 48.5 & 47.9 \\ \hline
50.9 & 50.7 & 51.8 & 54.9 & 50.5 & 50.7 & 51.0 & 52.5 & 51.3 & 50.7 \\ \hline
53.7 & 53.6 & 54.6 & 58.1 & 53.3 & 53.6 & 53.8 & 55.4 & 54.1 & 54.3 \\ \hline
56.6 & 56.3 & 57.5 & 61.1 & 56.2 & 56.5 & 56.7 & 58.4 & 57.1 & 59.1 \\ \hline
59.4 & 59.2 & 60.3 & 64.3 & 58.9 & 59.2 & 59.6 & 61.4 & 60.0 & 63.4 \\ 
\end{tabular}

\caption{Tempi in decelerazione}
\label{tab:dec}
\end{tabella}


	
	\clearpage
	\subsection{Grafici}
	Qui vengono riportati i grafici di ciascuna sessione di presa dati, con la relativa interpolazione lineare per ricavare $\alpha$ e $\beta$.
Per primi i grafici relativi alle accelerazioni. L'errore è abbastanza buono su questi, come si vede anche dagli stessi.
Si è posto in ascissa $\frac{t}{2}$ (dove si è diviso per due perché si
trovano le velocità nei punti medi), e in ordinata $\omega = \frac{2 n
\pi}{t}$, con $n$ numero dei giri

\begin{grafico}
    \centering
\input{../gnuplot/immagini/v0_grafico.tex}
\caption{Prima serie, accelerazione}
\label{fig:1}
\end{grafico}

\begin{grafico}
    \centering
\input{../gnuplot/immagini/v1_grafico.tex}
\caption{Seconda serie, accelerazione}
\label{fig:1}
\end{grafico}

\begin{grafico}
    \centering
\input{../gnuplot/immagini/v2_grafico.tex}
\caption{Terza serie, accelerazione}
\label{fig:1}
\end{grafico}

\begin{grafico}
    \centering
\input{../gnuplot/immagini/v3_grafico.tex}
\caption{Quarta serie, accelerazione}
\label{fig:1}
\end{grafico}

\begin{grafico}
    \centering
\input{../gnuplot/immagini/v4_grafico.tex}
\caption{Quinta serie, accelerazione}
\label{fig:1}
\end{grafico}

\begin{grafico}
    \centering
\input{../gnuplot/immagini/v5_grafico.tex}
\caption{Sesta serie, accelerazione}
\label{fig:1}
\end{grafico}

\begin{grafico}
    \centering
\input{../gnuplot/immagini/v6_grafico.tex}
\caption{Settima serie, accelerazione}
\label{fig:1}
\end{grafico}

\begin{grafico}
    \centering
\input{../gnuplot/immagini/v7_grafico.tex}
\caption{Ottava serie, accelerazione}
\label{fig:1}
\end{grafico}

\begin{grafico}
    \centering
\input{../gnuplot/immagini/v8_grafico.tex}
\caption{Nona serie, accelerazione}
\label{fig:1}
\end{grafico}

\begin{grafico}
    \centering
\input{../gnuplot/immagini/v9_grafico.tex}
\caption{Decima serie, accelerazione}
\label{fig:1}
\end{grafico}

Qui invece i dati in decelerazione.
L'errore in decelerazione è abbastanza grande, probabilmente perché l'intervallo di tempo era troppo limitato per osservare un effetto significativo (infatti le $\beta$ sono molto piccole rispetto alle $\alpha$)

\begin{grafico}
    \centering
\input{../gnuplot/immagini/v0r_grafico.tex}
\caption{Prima serie, decelerazione}
\label{fig:1}
\end{grafico}

\begin{grafico}
    \centering
\input{../gnuplot/immagini/v1r_grafico.tex}
\caption{Seconda serie, decelerazione}
\label{fig:1}
\end{grafico}

\begin{grafico}
    \centering
\input{../gnuplot/immagini/v2r_grafico.tex}
\caption{Terza serie, decelerazione}
\label{fig:1}
\end{grafico}

\begin{grafico}
    \centering
\input{../gnuplot/immagini/v3r_grafico.tex}
\caption{Quarta serie, decelerazione}
\label{fig:1}
\end{grafico}

\begin{grafico}
    \centering
\input{../gnuplot/immagini/v4r_grafico.tex}
\caption{Quinta serie, decelerazione}
\label{fig:1}
\end{grafico}

\begin{grafico}
    \centering
\input{../gnuplot/immagini/v5r_grafico.tex}
\caption{Sesta serie, decelerazione}
\label{fig:1}
\end{grafico}

\begin{grafico}
    \centering
\input{../gnuplot/immagini/v6r_grafico.tex}
\caption{Settima serie, decelerazione}
\label{fig:1}
\end{grafico}

\begin{grafico}
    \centering
\input{../gnuplot/immagini/v7r_grafico.tex}
\caption{Ottava serie, decelerazione}
\label{fig:1}
\end{grafico}

\begin{grafico}
    \centering
\input{../gnuplot/immagini/v8r_grafico.tex}
\caption{Nona serie, decelerazione}
\label{fig:1}
\end{grafico}

\begin{grafico}
    \centering
\input{../gnuplot/immagini/v9r_grafico.tex}
\caption{Decima serie, decelerazione}
\label{fig:1}
\end{grafico}


		
\section{Conclusioni}
	IL momento d'inerzia, calcolato con la seguente formula \[I_0 = \frac{R m g - m \alpha R^{2}}{\alpha - \beta} \] è risultato $0.27 \pm 0.01 Kgm^2$. QUesto risultato
è incompatibile con il valore atteso del volano (0.14). Icalcoli sembrano
corretti, probabilmente quindi c'è stato qualche errore nella raccolta dei dati.

	
\section{Codice}
	Qui ci sono i programmi usati per l'analisi dei dati

\begin{verbatim}
Propagazione errori:----------------------------------------------------

/*
 * errore_reciproco.cxx
 *
 *  Created on: Mar 31, 2014
 *      Author: andrea
 */
#include <iostream>

using namespace std;

int main()
{
	double err = 0.1;
	double val = 0;
	double err_rec = 0;
	int i = 1;
	while (cin >> val) {
		err_rec = i * 2 *3.14 * err / (val * val);
		cout << err_rec << endl;
		i++;
	}
	return 0;
}




Media Pesata:------------------------------------------------------------
#include <iostream>
#include <cmath>

using namespace std;
int main() {
	
	//long double media = 0;
	long double x;
	long double somma_x = 0;
	long double sigma_x = 0;
	long double somma_sigma2 = 0;
	long i = 0;
	while (cin >> x) {
		cin >> sigma_x;
		somma_x += x * 1 / (sigma_x * sigma_x);
		somma_sigma2 += 1 / (sigma_x * sigma_x);
		i++;
	}
	cout << somma_x / somma_sigma2 << endl;
	cout << sqrt(1 / somma_sigma2) << endl;
}
Calcolo momento d'inerzia e relativo errore:------------------------------------------------
#include <iostream>
#include <cmath>

using namespace std;

const long double r=0.01895, m=0.034, g=9.806, s_r=0.00001, s_g=0.001, s_m=0.0005;
double momento(double al, double be);
long double sigma(double alpha, double s_alpha, double beta, double s_beta);

int main(int argc, const char * argv[])
{
    
    //cin alpha beta ed errori
    long double alpha, beta, s_alpha, s_beta,I_0,s_I_0;
    
    cout << "---------------------Calcolare il momento d'inerzia---------------------";
    cout << endl;

    cout << "Alpha = ";
    cin  >> alpha;
    cout << "Errore di alpha = ";
    cin  >> s_alpha;
    cout << "Beta = ";
    cin  >> beta;
    cout << "Errore di beta = ";
    cin  >> s_beta;
    cout << endl;
    I_0   = momento(alpha, beta);
    s_I_0 = sigma(alpha, s_alpha, beta,s_beta);
    cout<<"Il momento è :  "<< I_0 << " ± " << s_I_0<< endl;
    return 0;
}

double momento(double al, double be){
    return ((r*m*g-m*al*r*r)/(al-be));
}

long double sigma(double alpha, double s_alpha, double beta, double s_beta){
    return (sqrt(
				 (((m*g-2*m*alpha*r)/(alpha-beta))   				 *   ((m*g-2*m*alpha*r)/(alpha-beta))    *s_r*s_r) +
                 (((r*g-r*alpha*r)/(alpha-beta))     				 *   ((r*g-r*alpha*r)/(alpha-beta))      *s_m*s_m) +
                 (((r*m-r*alpha*r*m)/(alpha-beta))   				 *   ((r*m-r*alpha*r*m)/(alpha-beta))    *s_g*s_g) +
                 (((r*r*m*beta-r*g*m)/((alpha-beta)*(alpha-beta)))   *   ((r*r*m*beta-r*g*m)/((alpha-beta)*(alpha-beta)))    *s_alpha*s_alpha) +
                 (((m*g*r-m*r*r*alpha)/((alpha-beta)*(alpha-beta)))  *   ((m*g*r-m*r*r*alpha)/((alpha-beta)*(alpha-beta)))   *s_beta*s_beta) ));
}

\end{verbatim}

	
%\subsection{Esempio immagini}
%\begin{figure}[p]
% \centering
% \includegraphics[width=0.8\textwidth]{spazio1}
% \caption{Spazio!}
% \label{fig:spazio1}
%\end{figure}


\end{document}
