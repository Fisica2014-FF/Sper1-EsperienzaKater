Si è inserito e avvolto il filo nel perno, facendo attenzione a non creare nodi, che potrebbero influire negativamente sulle misure.
Identificata una tacca sul volano per contare i giri, si è posizionato l'apparato e si è lasciato andare il peso. 
Uno di noi guardava il volano, con un dito sul pulsante del cronometro, mentre l'altro trascriveva i dati al computer. Dopo dodici giri il pesetto si staccava e il volano decelerava:
si lasciava trascorrere qualche giro mentre si resettava il cronometro (per non avere problemi di covarianza tra accelerazione e decelerazione) 
e si preparava il nuovo file di dati. Poi si ricominciava a misurare i tempi, abbiamo proceduto contando singoli giri perchè ci siamo resi conto
che altrimenti si perdeva facilmente il conto dei giri.
Essendo la decelerazione minore dell'accelerazione, in valore assoluto, si è
deciso di prendere dieci campioni ripetuti di 22 misure ciascuno. 
