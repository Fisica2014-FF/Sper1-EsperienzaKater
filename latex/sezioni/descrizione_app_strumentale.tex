In quest'esperienza si è utilizzato, come apparato sperimentale, un volano
(volano numero 7F). Esso è composto da un disco (di raggio $R = 0.01895 \pm
0.00001 Kg$ forma complessa) di una certa massa appeso al muro tramite un perno, e
collegato tramite un filo di massa trascurabile a un pesetto di massa $34 \pm 0.5 g$, di modo che quando il
pesetto scenda il volano possa acquistare velocità angolare.
Il filo è legato al disco interno tramite un foro in quest'ultimo, in cui
viene incastrata l'altra estremità del filo in questione, il quale, una volta
spiegato per intero, viene rimosso, per mezzo della forza peso, lasciando così girare "a vuoto" il volano.

