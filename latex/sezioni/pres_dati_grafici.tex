Qui vengono riportati i grafici di ciascuna sessione di presa dati, con la relativa interpolazione lineare per ricavare $\alpha$ e $\beta$.
Per primi i grafici relativi alle accelerazioni. L'errore è abbastanza buono su questi, come si vede anche dagli stessi.
Si è posto in ascissa $\frac{t}{2}$ (dove si è diviso per due perché si
trovano le velocità nei punti medi), e in ordinata $\omega = \frac{2 n
\pi}{t}$, con $n$ numero dei giri

\begin{grafico}
    \centering
\input{../gnuplot/immagini/v0_grafico.tex}
\caption{Prima serie, accelerazione}
\label{fig:1}
\end{grafico}

\begin{grafico}
    \centering
\input{../gnuplot/immagini/v1_grafico.tex}
\caption{Seconda serie, accelerazione}
\label{fig:1}
\end{grafico}

\begin{grafico}
    \centering
\input{../gnuplot/immagini/v2_grafico.tex}
\caption{Terza serie, accelerazione}
\label{fig:1}
\end{grafico}

\begin{grafico}
    \centering
\input{../gnuplot/immagini/v3_grafico.tex}
\caption{Quarta serie, accelerazione}
\label{fig:1}
\end{grafico}

\begin{grafico}
    \centering
\input{../gnuplot/immagini/v4_grafico.tex}
\caption{Quinta serie, accelerazione}
\label{fig:1}
\end{grafico}

\begin{grafico}
    \centering
\input{../gnuplot/immagini/v5_grafico.tex}
\caption{Sesta serie, accelerazione}
\label{fig:1}
\end{grafico}

\begin{grafico}
    \centering
\input{../gnuplot/immagini/v6_grafico.tex}
\caption{Settima serie, accelerazione}
\label{fig:1}
\end{grafico}

\begin{grafico}
    \centering
\input{../gnuplot/immagini/v7_grafico.tex}
\caption{Ottava serie, accelerazione}
\label{fig:1}
\end{grafico}

\begin{grafico}
    \centering
\input{../gnuplot/immagini/v8_grafico.tex}
\caption{Nona serie, accelerazione}
\label{fig:1}
\end{grafico}

\begin{grafico}
    \centering
\input{../gnuplot/immagini/v9_grafico.tex}
\caption{Decima serie, accelerazione}
\label{fig:1}
\end{grafico}

Qui invece i dati in decelerazione.
L'errore in decelerazione è abbastanza grande, probabilmente perché l'intervallo di tempo era troppo limitato per osservare un effetto significativo (infatti le $\beta$ sono molto piccole rispetto alle $\alpha$)

\begin{grafico}
    \centering
\input{../gnuplot/immagini/v0r_grafico.tex}
\caption{Prima serie, decelerazione}
\label{fig:1}
\end{grafico}

\begin{grafico}
    \centering
\input{../gnuplot/immagini/v1r_grafico.tex}
\caption{Seconda serie, decelerazione}
\label{fig:1}
\end{grafico}

\begin{grafico}
    \centering
\input{../gnuplot/immagini/v2r_grafico.tex}
\caption{Terza serie, decelerazione}
\label{fig:1}
\end{grafico}

\begin{grafico}
    \centering
\input{../gnuplot/immagini/v3r_grafico.tex}
\caption{Quarta serie, decelerazione}
\label{fig:1}
\end{grafico}

\begin{grafico}
    \centering
\input{../gnuplot/immagini/v4r_grafico.tex}
\caption{Quinta serie, decelerazione}
\label{fig:1}
\end{grafico}

\begin{grafico}
    \centering
\input{../gnuplot/immagini/v5r_grafico.tex}
\caption{Sesta serie, decelerazione}
\label{fig:1}
\end{grafico}

\begin{grafico}
    \centering
\input{../gnuplot/immagini/v6r_grafico.tex}
\caption{Settima serie, decelerazione}
\label{fig:1}
\end{grafico}

\begin{grafico}
    \centering
\input{../gnuplot/immagini/v7r_grafico.tex}
\caption{Ottava serie, decelerazione}
\label{fig:1}
\end{grafico}

\begin{grafico}
    \centering
\input{../gnuplot/immagini/v8r_grafico.tex}
\caption{Nona serie, decelerazione}
\label{fig:1}
\end{grafico}

\begin{grafico}
    \centering
\input{../gnuplot/immagini/v9r_grafico.tex}
\caption{Decima serie, decelerazione}
\label{fig:1}
\end{grafico}

